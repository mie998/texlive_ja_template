\documentclass[uplatex, twocolumn,10pt]{jsarticle}

\usepackage[dvipdfmx]{graphicx}
\usepackage{latexsym}
\usepackage{bmpsize}
\usepackage{url}
\usepackage{comment}

\def\Underline{\setbox0\hbox\bgroup\let\\\endUnderline}
\def\endUnderline{\vphantom{y}\egroup\smash{\underline{\box0}}\\}

\newcommand{\ttt}[1]{\texttt{#1}}

\begin{document}

\title{\bf{\LARGE{Sample of \LaTeX  Document} \\ \Large{\LaTeX のサンプルコード}}}
\author{西脇 圭亮\\京都大学}
\date{2021年9月12日(水)}
\maketitle


\section{はじめに}
\begin{figure}[t]
    \begin{center}
        \includegraphics[width=7cm]{images/syokuji_computer.png}
        \caption{パソコンの前でご飯を食べる人のイラスト}
        \label{fig:syokuji_computer}
    \end{center}
\end{figure}

このドキュメントはlatexによる文書作成のテストになります。
github に push することで github actions にて pdf が自動生成されます。
前回のコミットとの差分をとって diff.pdf を出力する機能も用意しています。
pdf の作成には actions/create-release を使用しているので、コミットにタグをつけてプッシュすることで作成できます。diff が欲しい場合はタグをつけてコミットしてください。

\begin{thebibliography}{99}
    \bibitem{irasutoya} いらすとや, last access 2019.6.13 \url{https://www.irasutoya.com/}



\end{thebibliography}

\end{document}
